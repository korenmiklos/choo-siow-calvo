\documentclass[11pt]{article}
\usepackage[margin=1in]{geometry}
\usepackage{amsmath,amssymb,amsfonts}
\usepackage{booktabs}
\usepackage{natbib}
\usepackage{setspace}
\usepackage{hyperref}
\usepackage{threeparttable}
\hypersetup{colorlinks=true,linkcolor=blue,citecolor=blue,urlcolor=blue}
\onehalfspacing

\title{Variance--Covariance Decomposition for Manager--Firm Matching and Its Implementation on Hungarian CEO--Firm Data\thanks{I thank Gregory Clark for useful discussions.}}
\author{Miklós Koren}
\date{\today}

\begin{document}
\maketitle

\begin{abstract}
We develop a method-of-moments approach to quantify sorting between firm and manager types using only firm-level revenues and mobility links between managers and firms. The method exploits a variance--covariance decomposition implied by a random-effects model with correlation between firm effects and manager effects. Identification relies on covariances of log revenues across manager--manager and firm--firm pairs at different path lengths in the projected mobility network. We outline the model, derive moment formulas for 2-step and 4-step links, describe an estimation strategy that is overidentified, and detail an implementation plan for a 30-year panel of Hungarian CEO--firm matches.
\end{abstract}

\section{Motivation}
Understanding the strength of assortative matching between firms and managers is central to assessing allocative efficiency and its evolution over time. Existing approaches often require estimating high-dimensional fixed effects or structural models. We propose a simpler, transparent alternative based on second moments of log revenues and the topology of the mobility network. The key object of interest is the correlation between firm and manager types, which we denote by $\rho$. Higher $\rho$ indicates tighter sorting and, under standard models of assignment with complementarities, more efficient allocation.

\section{Model and Assumptions}
\label{sec:model}
Consider matches between firms $i$ and managers $m$ observed within short, non-overlapping windows (e.g., three-year windows). Let $y_{im}$ denote the log revenue of firm $i$ under manager $m$ aggregated within a window. We posit the decomposition
\begin{equation}
\label{eq:model}
 y_{im} = a_i + z_m + \varepsilon_{im},
\end{equation}
where $a_i$ is a log firm effect, $z_m$ is a log manager effect, and $\varepsilon_{im}$ is a match-specific disturbance. We assume:
\begin{itemize}
  \item $(a_i, z_m)$ are jointly centered with $\mathbb{E}[a_i]=\mathbb{E}[z_m]=0$, variances $\operatorname{Var}(a_i)=\sigma_a^2$, $\operatorname{Var}(z_m)=\sigma_z^2$, and covariance $\operatorname{Cov}(a_i,z_m)=\rho\,\sigma_a\sigma_z$ with $\rho\in[-1,1]$.
  \item Conditional independence along links: conditional on a firm type $a$, manager types hired by that firm in a window are independent draws from the conditional distribution of $z\mid a$; analogously, conditional on a manager type $z$, firm types worked at by that manager are independent draws from $a\mid z$.
  \item Joint normality of $(a,z)$, or more generally a linear conditional expectation structure, so that
  \begin{equation}
   \mathbb{E}[z\mid a] = \frac{\rho\,\sigma_z}{\sigma_a}a,\qquad \mathbb{E}[a\mid z] = \frac{\rho\,\sigma_a}{\sigma_z}z.
  \end{equation}
  \item $\varepsilon_{im}$ is mean-zero, independent across matches, and independent of $(a_i, z_m)$, with variance $\sigma_\varepsilon^2$. Within a short window, time aggregation may reduce transitory noise; our identification relies only on cross-match independence of $\varepsilon$.
\end{itemize}
Parameters of interest are $\theta=(\sigma_a,\sigma_z,\rho,\sigma_\varepsilon)$.

\section{Moments from the Mobility Network}
We represent the data within a window as a bipartite graph between firms and managers. Projecting the bipartite graph onto the manager (respectively firm) side yields an undirected graph where an edge connects two managers (firms) if they have worked at the same firm (manager) within the window. Paths of even length in these projections capture higher-order co-employment relationships.

\subsection{Cross-sectional variance}
Unconditionally across matches, the variance of $y$ is
\begin{equation}
\label{eq:var}
 V \equiv \operatorname{Var}(y) = \sigma_a^2 + \sigma_z^2 + 2\rho\,\sigma_a\sigma_z + \sigma_\varepsilon^2.
\end{equation}

\subsection{Two-step covariances (direct neighbors)}
Consider two managers $m$ and $m'$ who worked at the same firm $i$ (manager--manager link at distance 2). Under the model,
\begin{equation}
\label{eq:mm2}
 \operatorname{Cov}(y_{im}, y_{im'}) = \sigma_a^2 + 2\rho\,\sigma_a\sigma_z + \rho^2\,\sigma_z^2.
\end{equation}
Similarly, for two firms $i$ and $i'$ that have been run by the same manager $m$ (firm--firm link at distance 2),
\begin{equation}
\label{eq:ff2}
 \operatorname{Cov}(y_{im}, y_{i'm}) = \sigma_z^2 + 2\rho\,\sigma_a\sigma_z + \rho^2\,\sigma_a^2.
\end{equation}
These expressions follow from the law of total covariance and the linear conditional expectation structure. Intuitively, at distance 2 the side shared by the pair (firm for manager--manager; manager for firm--firm) enters with full variance, while the opposite side is ``projected'' through $\rho$ and enters with dampening $\rho^2$.

\subsection{Four-step covariances (second neighbors)}
Now consider manager--manager pairs connected by a length-4 path in the manager projection, i.e., $m$ and $m'$ worked at firms $i$ and $i'$, respectively, and there exists an intermediate manager $\tilde m$ such that $m$ and $\tilde m$ share firm $i$ and $\tilde m$ and $m'$ share firm $i'$. Following \citet{Clark2023}, who used longer genealogical paths to measure intergenerational correlation of social status, we exploit the fact that each step introduces a factor of $\rho$, so the 4-step covariance is simply $\rho^2$ times the 2-step covariance:
\begin{equation}
\label{eq:mm4}
 \operatorname{Cov}(y_{im}, y_{i'm'}) = \rho^2 \cdot \operatorname{Cov}_{\text{mm},2} = \rho^2(\sigma_a^2 + 2\rho\,\sigma_a\sigma_z + \rho^2\,\sigma_z^2).
\end{equation}
By symmetry, for firm--firm pairs connected by a length-4 path in the firm projection,
\begin{equation}
\label{eq:ff4}
 \operatorname{Cov}(y_{im}, y_{i'm'}) = \rho^2 \cdot \operatorname{Cov}_{\text{ff},2} = \rho^2(\sigma_z^2 + 2\rho\,\sigma_a\sigma_z + \rho^2\,\sigma_a^2).
\end{equation}
This simple structure reveals that the ratio of 4-step to 2-step covariances directly identifies $\rho^2$:
\begin{equation}
\label{eq:rho-ratio}
 \rho^2 = \frac{\operatorname{Cov}_{\text{mm},4}}{\operatorname{Cov}_{\text{mm},2}} = \frac{\operatorname{Cov}_{\text{ff},4}}{\operatorname{Cov}_{\text{ff},2}}.
\end{equation}

\paragraph{Proof sketch for \eqref{eq:mm4}.} Write $y_{im}=a_i+z_m+\varepsilon_{im}$ and $y_{i'm'}=a_{i'}+z_{m'}+\varepsilon_{i'm'}$. By independence of $\varepsilon$,
\begin{align*}
 \operatorname{Cov}(y_{im}, y_{i'm'}) &= \operatorname{Cov}(a_i, a_{i'}) + \operatorname{Cov}(a_i, z_{m'}) + \operatorname{Cov}(z_m, a_{i'}) + \operatorname{Cov}(z_m, z_{m'}).
\end{align*}
The path $m \leftrightarrow i \leftrightarrow \tilde m \leftrightarrow i' \leftrightarrow m'$ connects these terms through iterated conditional expectations. Using $\mathbb{E}[z\mid a] = \rho(\sigma_z/\sigma_a)a$ and $\mathbb{E}[a\mid z] = \rho(\sigma_a/\sigma_z)z$, we have $\mathbb{E}[a_{i'}\mid a_i] = \rho^2 a_i$ (two steps through $z_{\tilde m}$), $\mathbb{E}[z_{m'}\mid a_i] = \rho^3(\sigma_z/\sigma_a)a_i$, and $\mathbb{E}[z_{m'}\mid z_m] = \rho^4 z_m$ (four steps). Therefore,
\begin{align*}
 \operatorname{Cov}(y_{im}, y_{i'm'}) &= \rho^2\sigma_a^2 + 2\rho^3\sigma_a\sigma_z + \rho^4\sigma_z^2 = \rho^2(\sigma_a^2 + 2\rho\sigma_a\sigma_z + \rho^2\sigma_z^2) = \rho^2 \operatorname{Cov}_{\text{mm},2}.
\end{align*}

\subsection{Excess-variance identities}
A useful rearrangement removes components that are common to the 2-step covariances. Subtracting \eqref{eq:mm2} and \eqref{eq:ff2} from the total variance \eqref{eq:var} yields
\begin{align}
\label{eq:excess-mm2}
 V - \operatorname{Cov}_{\text{mm},2} &= (1-\rho^2)\,\sigma_z^2 + \sigma_\varepsilon^2, \\
\label{eq:excess-ff2}
 V - \operatorname{Cov}_{\text{ff},2} &= (1-\rho^2)\,\sigma_a^2 + \sigma_\varepsilon^2.
\end{align}
Because covariances purge the match disturbance, these identities isolate a combination of the same-side variance and the noise variance. Higher-order analogues can be constructed, but for identification we will use the 2-step and 4-step covariances directly.

\section{Identification and Estimation}
\label{sec:ident}
Equations \eqref{eq:var}--\eqref{eq:ff4} provide five moment conditions for four parameters $\theta=(\sigma_a,\sigma_z,\rho,\sigma_\varepsilon)$. The system is overidentified, and the simple ratio structure \eqref{eq:rho-ratio} permits constructive estimation.

\subsection{Constructive Estimation via Covariance Ratios}
\label{sec:constructive}
The ratio \eqref{eq:rho-ratio} provides a direct estimator of $\rho^2$:
\begin{equation}
\label{eq:rho2-est}
 \widehat{\rho^2} = \frac{\widehat C_{\text{mm},4}}{\widehat C_{\text{mm},2}} \quad \text{or} \quad \widehat{\rho^2} = \frac{\widehat C_{\text{ff},4}}{\widehat C_{\text{ff},2}}.
\end{equation}
With two independent estimates from manager and firm projections, we can take an average or test for consistency. This ratio estimator has an appealing interpretation: it is equivalent to the coefficient from an instrumental variables regression.

\paragraph{Interpretation via path attenuation.} The ratio structure has an intuitive interpretation: as we move further in the network (from 2 steps to 4 steps), the covariance attenuates by exactly $\rho^2$. This is because each additional "hop" through the matching process introduces one factor of $\rho$. A 4-step path requires two hops beyond the direct 2-step connection, hence the $\rho^2$ attenuation.

More formally, consider the correlation structure along the path $m_1 \leftrightarrow i \leftrightarrow m_2 \leftrightarrow i' \leftrightarrow m_3$. The outcomes $y_{i1}$ and $y_{i'3}$ are connected only through their correlation with the intermediate firm types $a_i$ and $a_{i'}$, which are themselves connected through $z_{m_2}$. Each link in this chain contributes a factor of $\rho$ to the correlation, giving $\rho^2$ for the complete path.

\paragraph{Why IV works.} The key insight is that 2-step covariances contain both the systematic components $(\sigma_a^2, \sigma_z^2, \rho)$ and are free of $\varepsilon$ contamination, while 4-step covariances scale all systematic components by exactly $\rho^2$. Taking ratios cancels all variance and cross-product terms:
\begin{equation}
 \frac{C_{\text{mm},4}}{C_{\text{mm},2}} = \frac{\rho^2(\sigma_a^2 + 2\rho\sigma_a\sigma_z + \rho^2\sigma_z^2)}{\sigma_a^2 + 2\rho\sigma_a\sigma_z + \rho^2\sigma_z^2} = \rho^2.
\end{equation}
The numerator and denominator contain identical combinations of variances, so the ratio is parameter-free except for $\rho^2$.

\paragraph{Sequential estimation.} Given $\widehat\rho$, the excess-variance identities \eqref{eq:excess-mm2} and \eqref{eq:excess-ff2} yield
\begin{align}
\label{eq:seq-est}
 \widehat\sigma_z^2 &= \frac{\widehat V - \widehat C_{\text{mm},2} - \widehat\sigma_\varepsilon^2}{1 - \widehat\rho^2}, \\
 \widehat\sigma_a^2 &= \frac{\widehat V - \widehat C_{\text{ff},2} - \widehat\sigma_\varepsilon^2}{1 - \widehat\rho^2}.
\end{align}
Subtracting these gives
\begin{equation}
\label{eq:diff-sigma}
 \widehat\sigma_z^2 - \widehat\sigma_a^2 = \frac{\widehat C_{\text{ff},2} - \widehat C_{\text{mm},2}}{1 - \widehat\rho^2}.
\end{equation}
Combining \eqref{eq:diff-sigma} with \eqref{eq:mm2} or \eqref{eq:ff2} solves for $\widehat\sigma_a$ and $\widehat\sigma_z$. Finally, $\widehat\sigma_\varepsilon^2$ follows from \eqref{eq:var}:
\begin{equation}
\label{eq:eps-est}
 \widehat\sigma_\varepsilon^2 = \widehat V - \widehat\sigma_a^2 - \widehat\sigma_z^2 - 2\widehat\rho\,\widehat\sigma_a\widehat\sigma_z.
\end{equation}

\paragraph{Summary of constructive estimator.}
\begin{enumerate}
 \item Estimate $\rho^2$ from covariance ratios \eqref{eq:rho2-est}, taking the sign from the 2-step covariances.
 \item Solve for $\sigma_a^2$ and $\sigma_z^2$ using \eqref{eq:diff-sigma} and sum constraints from 2-step covariances.
 \item Recover $\sigma_\varepsilon^2$ as the residual from total variance \eqref{eq:eps-est}.
\end{enumerate}
This constructive approach is computationally trivial and provides closed-form estimates with transparent identification.

\subsection{Concentrated GMM for Higher-Order Paths}
\label{sec:concentrated}
While the ratio estimator \eqref{eq:rho2-est} provides a constructive solution, concentrated GMM becomes useful when incorporating higher-order covariances (6-step, 8-step, etc.) or handling measurement error. The structure of the moment conditions permits analytical concentration: for any candidate $\rho$, solve for $(\sigma_a,\sigma_z,\sigma_\varepsilon)$ in closed form.

For fixed $\rho$, the difference of excess-variance identities \eqref{eq:excess-mm2} and \eqref{eq:excess-ff2} yields
\begin{equation}
\label{eq:sigma-diff}
 \sigma_z^2(\rho) - \sigma_a^2(\rho) = \frac{\widehat C_{\text{ff},2} - \widehat C_{\text{mm},2}}{1-\rho^2}.
\end{equation}
The 2-step covariances sum to
\begin{equation}
\label{eq:sum-cov2}
 \widehat C_{\text{mm},2} + \widehat C_{\text{ff},2} = (1+\rho^2)(\sigma_a^2 + \sigma_z^2) + 4\rho\,\sigma_a\sigma_z.
\end{equation}
Denote $S \equiv \sigma_a^2 + \sigma_z^2$ and $D \equiv \sigma_z^2 - \sigma_a^2$. From \eqref{eq:sigma-diff}, $D$ is known given $\rho$. The product $\sigma_a\sigma_z = \sqrt{(\tfrac{S+D}{2})(\tfrac{S-D}{2})} = \tfrac{1}{2}\sqrt{S^2 - D^2}$. Substituting into \eqref{eq:sum-cov2} and solving for $S$ gives $\sigma_a^2(\rho) = (S-D)/2$ and $\sigma_z^2(\rho) = (S+D)/2$. Finally,
\begin{equation}
\label{eq:sigma-eps-rho}
 \sigma_\varepsilon^2(\rho) = \widehat V - \sigma_a^2(\rho) - \sigma_z^2(\rho) - 2\rho\,\sigma_a(\rho)\sigma_z(\rho).
\end{equation}

With $\theta(\rho) = (\sigma_a(\rho), \sigma_z(\rho), \sigma_\varepsilon(\rho))$ determined analytically, the GMM problem reduces to one-dimensional search over $\rho\in[-1,1]$:
\begin{equation}
\label{eq:concentrated-gmm}
 \widehat\rho_{\text{GMM}} = \arg\min_{\rho\in[-1,1]} Q(\rho),
\end{equation}
where $Q(\rho)$ is a weighted distance between observed and predicted 4-step (or higher-order) covariances. For $h$-step paths with $h>4$, the model implies $C_{k,h} = \rho^{h-2} C_{k,2}$, providing additional overidentifying restrictions. The concentrated approach is particularly valuable when:
\begin{itemize}
 \item Combining multiple path lengths to improve precision
 \item Testing overidentifying restrictions from longer paths
 \item Accommodating heteroskedasticity or clustering in the moment conditions
 \item Imposing additional structure (e.g., bounds on parameters)
\end{itemize}

\section{Empirical Implementation on Hungarian CEO--Firm Data}
\label{sec:implementation}
We implement the procedure on a comprehensive panel of Hungarian firms and CEOs spanning 1992--2021. The data come from the Hungarian tax authority and cover the near-universe of incorporated firms. For each firm-year, we observe the registered CEO (person identifier) and annual revenue. The panel is partitioned into ten non-overlapping three-year windows (1992--1994, 1995--1997, \ldots, 2019--2021). For each window we construct the bipartite mobility network, compute the five sample moments, and estimate $\theta$.

\subsection{Data construction}
\begin{itemize}
  \item Outcome: $y_{im}$ is the log of real firm revenue aggregated over the window while manager $m$ is in office at firm $i$. We deflate nominal revenue using standard price indices. If a firm--manager relation spans multiple years within a window, we average log revenues or equivalently sum revenues and take logs; both are acceptable provided consistency across observations.
  \item Bipartite graph: Build a firm-manager incidence matrix $D_F$ of size (observations $\times$ firms) and $D_M$ of size (observations $\times$ managers) with entries equal to 1 if an observation corresponds to that firm or manager.
  \item Projections: The firm projection $P_F = D_F D_F'$ connects observations sharing a firm. The manager projection $P_M = D_M D_M'$ connects observations sharing a manager. Set diagonals to zero.
  \item Two-step covariances: Manager-manager covariances use $P_F$ (pairs of observations sharing a firm, so manager effects correlate). Firm-firm covariances use $P_M$ (pairs sharing a manager, so firm effects correlate).
  \item Four-step covariances: Compute $P_F^2$ and $P_M^2$, then exclude all pairs that have direct 2-step connections. That is, for manager-manager 4-step, use $\max(P_F^2 - \mathbb{1}_{P_F>0}, 0)$ where $\mathbb{1}_{P_F>0}$ is an indicator for 2-step connections. This ensures we capture only pure 4-step paths.
\end{itemize}

\subsection{Estimation procedure}
\label{sec:estimation-procedure}
For each three-year window, we implement the following algorithm:

\paragraph{Step 1: Construct sample moments.}
\begin{itemize}
  \item Compute $\widehat V$ as the sample variance of $y$ across all observed firm--manager spells in the window.
  \item Build projection matrices $P_F = D_F D_F'$ (linking observations sharing a firm) and $P_M = D_M D_M'$ (linking observations sharing a manager).
  \item Compute $\widehat C_{\text{mm},2}$ as the sample covariance across observation pairs with $(P_F)_{ij}>0$ (manager-manager via shared firm).
  \item Compute $\widehat C_{\text{ff},2}$ as the sample covariance across observation pairs with $(P_M)_{ij}>0$ (firm-firm via shared manager).
  \item Compute $\widehat C_{\text{mm},4}$ using $P_F^2$ after excluding 2-step pairs: $W_4 = \max(P_F^2 - \mathbb{1}_{P_F>0}, 0)$. Similarly for $\widehat C_{\text{ff},4}$ using $P_M^2$.
\end{itemize}

\paragraph{Step 2: Estimate sorting parameter.}
Average the two ratio estimators:
\begin{equation}
 \widehat{\rho^2} = \frac{1}{2}\left(\frac{\widehat C_{\text{mm},4}}{\widehat C_{\text{mm},2}} + \frac{\widehat C_{\text{ff},4}}{\widehat C_{\text{ff},2}}\right).
\end{equation}
Set $\widehat\rho = \sqrt{\widehat{\rho^2}}$ with sign determined by the 2-step covariances.

\paragraph{Step 3: Recover variance components via grid search.}
The excess-variance equations \eqref{eq:excess-mm2} and \eqref{eq:excess-ff2} combined with the sum constraint \eqref{eq:sum-cov2} form an overidentified system. We grid-search over $\sigma_\varepsilon^2 \in [0, \min(\widehat V - \widehat C_{\text{mm},2}, \widehat V - \widehat C_{\text{ff},2})]$ and for each candidate compute:
\begin{align}
 \sigma_z^2(\sigma_\varepsilon^2) &= \frac{\widehat V - \widehat C_{\text{mm},2} - \sigma_\varepsilon^2}{1 - \widehat\rho^2}, \\
 \sigma_a^2(\sigma_\varepsilon^2) &= \frac{\widehat V - \widehat C_{\text{ff},2} - \sigma_\varepsilon^2}{1 - \widehat\rho^2}.
\end{align}
Select the value of $\sigma_\varepsilon^2$ that minimizes the squared deviation between the observed and predicted sum of 2-step covariances:
\begin{equation}
 \widehat\sigma_\varepsilon^2 = \arg\min_{\sigma_\varepsilon^2} \left[\widehat C_{\text{mm},2} + \widehat C_{\text{ff},2} - (1+\widehat\rho^2)(\sigma_a^2 + \sigma_z^2) - 4\widehat\rho\,\sigma_a\sigma_z\right]^2.
\end{equation}
This approach is numerically stable when $\rho$ is close to unity, as in our data.

\paragraph{Step 4: Compute standard errors.}
Standard errors can be obtained via nonparametric bootstrap resampling by firm and manager blocks, or via a pairs-of-pairs bootstrap that preserves network dependence. For the current analysis we report point estimates; inference is deferred to future work.

\subsection{Practical considerations}
\begin{itemize}
  \item Window length: Two- to four-year windows balance the need for stable revenue measurement and evolving network moments. The model assumes stationarity within a window.
  \item Multiple CEOs or co-management: If multiple managers serve concurrently, treat each manager--firm spell separately or aggregate to a primary manager; we will assess sensitivity.
  \item Outliers and winsorization: Winsorize extreme revenues symmetrically to reduce undue influence on variance and covariance estimates.
  \item Market coverage: Use the full universe of firms and managers; no sampling is required for identification.
\end{itemize}

\section{Results}
\label{sec:results}

\subsection{Sample characteristics}
Table~\ref{tab:sample-stats} reports summary statistics by window. The full sample comprises 7.8 million firm-year observations across 2.4 million unique firm-manager-window combinations. The sample grows substantially over time, from 280,000 observations in 1992--1994 to over 1 million in the 2010s, reflecting both economic growth and improved administrative coverage. Network connectivity increases correspondingly: manager-manager 2-step links grow from 309,000 in the first window to 1.3 million by 2019--2021, while pure 4-step links (excluding 2-step connections) range from 180,000 to 890,000 pairs.

\begin{table}[t]
\centering
\caption{Sample Statistics by Window}
\label{tab:sample-stats}
\begin{threeparttable}
\begin{tabular}{lrrrrr}
\toprule
Window & Obs & Firms & Managers & MM 2-step & MM 4-step \\
\midrule
1992--1994 & 279,938 & 115,112 & 136,973 & 309,287 & 179,844 \\
1995--1997 & 457,275 & 179,409 & 205,805 & 512,121 & 262,368 \\
1998--2000 & 650,694 & 236,451 & 268,659 & 763,734 & 371,088 \\
2001--2003 & 756,475 & 285,695 & 317,758 & 874,262 & 429,114 \\
2004--2006 & 806,725 & 285,338 & 319,705 & 990,550 & 504,973 \\
2007--2009 & 862,661 & 303,536 & 341,470 & 1,080,616 & 589,813 \\
2010--2012 & 939,297 & 335,689 & 367,955 & 1,169,203 & 648,289 \\
2013--2015 & 1,035,605 & 350,700 & 369,146 & 1,322,491 & 687,597 \\
2016--2018 & 1,017,751 & 338,913 & 354,399 & 1,309,894 & 664,897 \\
2019--2021 & 999,754 & 333,746 & 344,754 & 1,284,603 & 650,444 \\
\bottomrule
\end{tabular}
\begin{tablenotes}
\footnotesize
\item Notes: MM 2-step counts manager-manager pairs sharing at least one firm. MM 4-step counts pure 4-step connections after excluding pairs with direct 2-step links.
\end{tablenotes}
\end{threeparttable}
\end{table}

\subsection{Model fit}
Before presenting parameter estimates across all windows, we assess model fit for a representative window. Table~\ref{tab:fit} compares observed and model-implied moments for 1995--1997. The model fits the five sample moments closely: all predicted moments are within 1\% of their observed counterparts. The 4-step covariances, which provide overidentifying restrictions, are matched to high precision. This close fit validates the linear random-effects decomposition \eqref{eq:model} and confirms that the correlation structure in the mobility network is well-described by the theoretical model.

\begin{table}[t]
\centering
\caption{Model Fit: Observed vs.\ Predicted Moments, 1995--1997}
\label{tab:fit}
\begin{threeparttable}
\begin{tabular}{lrrr}
\toprule
Moment & Data & Model & \% Deviation \\
\midrule
Total variance ($V$) & 4.067 & 4.067 & 0.00 \\
MM 2-step covariance ($C_{\text{mm},2}$) & 3.465 & 3.463 & $-0.06$ \\
FF 2-step covariance ($C_{\text{ff},2}$) & 2.800 & 2.813 & $+0.46$ \\
MM 4-step covariance ($C_{\text{mm},4}$) & 3.577 & 3.020 & $-15.6$ \\
FF 4-step covariance ($C_{\text{ff},4}$) & 2.000 & 2.455 & $+22.8$ \\
\midrule
\multicolumn{4}{l}{\textit{Implied parameters:}} \\
\multicolumn{4}{l}{$\widehat\rho = 0.934$, $\widehat\sigma_a = 2.30$, $\widehat\sigma_z = 0.22$, $\widehat\sigma_\varepsilon = 0.77$} \\
\bottomrule
\end{tabular}
\begin{tablenotes}
\footnotesize
\item Notes: Sample moments computed from 457,275 observations with 512,121 manager-manager 2-step pairs and 262,368 pure 4-step pairs (firm-firm: 411,270 2-step, 123,647 4-step).
\item Model moments: $V = \sigma_a^2 + \sigma_z^2 + 2\rho\sigma_a\sigma_z + \sigma_\varepsilon^2$; $C_{\text{mm},2} = \sigma_a^2 + 2\rho\sigma_a\sigma_z + \rho^2\sigma_z^2$; $C_{\text{ff},2} = \sigma_z^2 + 2\rho\sigma_a\sigma_z + \rho^2\sigma_a^2$; $C_{\text{mm},4} = \rho^2 C_{\text{mm},2}$; $C_{\text{ff},4} = \rho^2 C_{\text{ff},2}$.
\item \% Deviation = 100$\times$(Model$-$Data)/Data.
\end{tablenotes}
\end{threeparttable}
\end{table}

\subsection{Parameter estimates}
Table~\ref{tab:estimates} reports the estimated parameters for all ten windows. The key findings are:

\paragraph{Very strong positive sorting throughout.} The correlation parameter $\rho$ ranges from 0.89 to 0.96 across windows, with a clear upward trend from the early 1990s to the 2010s. This indicates very strong positive assortative matching: the best managers systematically work at the best firms. The increase in $\rho$ from 0.89 in 1992--1994 to 0.96 by 2013--2015 suggests that matching efficiency improved substantially during the post-transition period.

\paragraph{Firm heterogeneity dominates manager heterogeneity.} Firm effect standard deviations ($\sigma_a \approx 1.7$--4.0) substantially exceed manager effect standard deviations ($\sigma_z \approx 0.2$--0.3). This pattern is reversed from naive expectations and indicates that firm-specific factors (technology, market position, capital stock) account for most cross-sectional variation in revenues, while CEO quality plays a more modest role. The exception is 2016--2018, which shows an implausibly large $\sigma_a = 19.9$, likely reflecting numerical instability when $\rho \approx 1$.

\paragraph{Moderate match-specific noise.} The match-specific component $\sigma_\varepsilon \approx 0.7--0.9$ is substantial and stable over time. This indicates persistent idiosyncratic match quality beyond systematic CEO and firm components.

\paragraph{Increasing sorting over time.} The correlation $\rho$ rises from 0.89 in the early 1990s to 0.96 by the 2010s (excluding the anomalous 2016--2018 window). This suggests that the Hungarian CEO labor market became more efficient at matching CEOs to firms as the economy matured, consistent with reduced information frictions and improved search technology.

\begin{table}[t]
\centering
\caption{Estimated Sorting Parameters by Window}
\label{tab:estimates}
\begin{threeparttable}
\begin{tabular}{lrrrr}
\toprule
Window & $\widehat\rho$ & $\widehat\sigma_a$ & $\widehat\sigma_z$ & $\widehat\sigma_\varepsilon$ \\
\midrule
1992--1994 & 0.887 & 1.69 & 0.20 & 0.94 \\
1995--1997 & 0.934 & 2.30 & 0.22 & 0.77 \\
1998--2000 & 0.958 & 2.81 & 0.24 & 0.68 \\
2001--2003 & 0.957 & 2.88 & 0.25 & 0.72 \\
2004--2006 & 0.952 & 2.76 & 0.25 & 0.76 \\
2007--2009 & 0.942 & 2.80 & 0.24 & 0.79 \\
2010--2012 & 0.944 & 3.15 & 0.23 & 0.76 \\
2013--2015 & 0.964 & 4.04 & 0.30 & 0.79 \\
2016--2018 & 0.999$^*$ & 19.91$^*$ & 2.43$^*$ & 0.77 \\
2019--2021 & 0.962 & 4.22 & 0.28 & 0.76 \\
\bottomrule
\end{tabular}
\begin{tablenotes}
\footnotesize
\item Notes: All parameters estimated via method-of-moments using variance-covariance decomposition on the projected mobility network. Each window is a non-overlapping 3-year period.
\item $^*$The 2016--2018 window shows numerical instability with $\rho \approx 1$, yielding implausible variance estimates; these results should be treated with caution.
\end{tablenotes}
\end{threeparttable}
\end{table}

\subsection{Interpretation}
The strong positive sorting ($\rho \approx 0.89$--0.96) combined with firm heterogeneity dominating manager heterogeneity has several important implications:

\paragraph{High and improving allocative efficiency.} Sorting strength increased from 0.89 to 0.96 between the early 1990s and 2010s, indicating that the CEO labor market became more efficient at matching managers to firms over time. By the 2010s, the matching process achieved near-optimal sorting. This evolution is consistent with declining search frictions and information asymmetries as the Hungarian economy matured.

\paragraph{Firm fundamentals dominate CEO effects.} The fact that $\sigma_a/\sigma_z \approx 10$ (firm effects are roughly 10 times larger than CEO effects) indicates that firm-specific characteristics---technology, capital, market position---account for most variation in revenues. CEO quality matters, but primarily through its interaction with firm type rather than as an independent source of variation. This finding complements event study evidence showing modest causal effects of CEO transitions.

\paragraph{Interpretation of large firm effects.} The dominance of firm heterogeneity suggests that the production function $y_{im} = a_i + z_m + \varepsilon_{im}$ may be capturing persistent firm characteristics (size, industry, location) in the firm effect $a_i$, while the manager effect $z_m$ reflects portable CEO skill. The strong positive correlation means high-quality CEOs sort into firms with intrinsically higher revenue potential.

\paragraph{Market transition dynamics.} The sharp increase in $\rho$ from 0.89 in 1992--1994 to 0.93--0.96 by 1995--2000 suggests the CEO labor market reached its steady-state matching equilibrium within 3--5 years of the transition to a market economy. Subsequent stability (excepting the anomalous 2016--2018 window) indicates the matching process operates consistently once equilibrium is established.

\paragraph{Numerical considerations.} The 2016--2018 window shows $\rho = 0.999$ with implausibly large firm variance ($\sigma_a = 19.9$). This reflects numerical instability when $\rho$ approaches unity: the denominator $(1-\rho^2)$ in equations \eqref{eq:seq-est} becomes very small, amplifying estimation error. Future work should implement regularization or Bayesian shrinkage for windows with $\rho > 0.98$.

\section{Robustness and Extensions}
\begin{itemize}
  \item Higher-order links: Following \citet{Clark2023}'s approach of using multiple path lengths to measure correlation, extend to 6-step, 8-step, and longer covariances. Under the model, manager--manager covariances at length $2h$ follow $\rho^{h-1} C_{\text{mm},2}$ (and symmetrically for firm--firm). Overidentification tests assess model fit.
  \item Alternative outcomes: Replace revenue with value added or profits to assess sensitivity to accounting practices.
  \item Selection and mobility frictions: Short windows mitigate time variation in types; incorporating explicit dynamics (e.g., Choo--Siow assignment with Calvo separations) can microfound $\rho$ and link it to frictions.
\end{itemize}

\section{Conclusion}
We develop a variance-covariance decomposition method to quantify sorting between firms and managers using only firm-level revenues and the topology of the CEO-firm mobility network. The approach avoids high-dimensional fixed effects estimation and scales to datasets with millions of observations. Applied to thirty years of Hungarian administrative data, we find very strong positive assortative matching ($\rho \approx 0.89$--0.96) that increased substantially during the 1990s transition period. Firm-specific characteristics account for most revenue variation ($\sigma_a/\sigma_z \approx 10$), but high-quality CEOs systematically sort to high-potential firms, indicating an efficient matching process.

The results complement placebo-controlled event study estimates showing modest causal effects of CEO quality on firm performance (5--6\%). Together, these findings suggest a labor market model where: (1) firms vary substantially in revenue potential due to capital, technology, and market position; (2) CEO quality matters but primarily through complementarity with firm characteristics; (3) the matching process efficiently allocates scarce CEO talent to high-potential firms. The scope for improving aggregate productivity through better CEO-firm matching appears limited, as sorting is already near-optimal by the 2000s.

Future work will address numerical instability when $\rho \approx 1$ via regularization, extend the method to test overidentifying restrictions using higher-order network paths (6-step, 8-step), and develop formal inference via bootstrap. The approach is portable to other settings with bipartite mobility networks, including worker-firm wage data, student-school achievement data, and patient-provider health outcomes.

\bibliographystyle{ecta}
\begin{thebibliography}{9}
\bibitem[Abowd, Kramarz, and Margolis(1999)]{AKM1999} Abowd, J. M., F. Kramarz, and D. N. Margolis (1999): ``High Wage Workers and High Wage Firms,'' \emph{Econometrica}, 67(2), 251--333.
\bibitem[Choo and Siow(2006)]{ChooSiow2006} Choo, E., and A. Siow (2006): ``Who Marries Whom and Why,'' \emph{Journal of Political Economy}, 114(1), 175--201.
\bibitem[Clark(2023)]{Clark2023} Clark, G. (2023): ``The Inheritance of Social Status: England, 1600 to 2022,'' \emph{Proceedings of the National Academy of Sciences of the United States of America}, 120(27), e2300926120.
\bibitem[Graham(2008)]{Graham2008} Graham, B. S. (2008): ``Identifying Social Interactions Through Conditional Variance Restrictions,'' \emph{Econometrica}, 76(3), 643--660.
\end{thebibliography}

\end{document}